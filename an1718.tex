\documentclass[french]{article}
% An example how to use the calendar library and modify the layout, i.e. put
% Sunday as the first week day.
%
% Author: Berteun Damman
% et mes ajustement (rentrée 2014)
% version juillet-2015 ; août-2016 ; sept-2017

\usepackage[french]{babel}
\usepackage[utf8]{inputenc}
\usepackage[T1]{fontenc}

\usepackage[french]{translator}% À mettre avant tikz-calendar

\usepackage{tikz}
\usetikzlibrary{calendar}

%%%<
\usepackage{verbatim}
\usepackage[active,tightpage]{preview}
\PreviewEnvironment{tikzpicture}
\setlength\PreviewBorder{5pt}%
%%>

\begin{comment}
:Title: Changing the default calendar layout
:Tags: Calendars; Matrices
:Author: Berteun Damman

An example how to modify the calendar drawing code so Sundays are drawn
as the first day of the week; it also shows how to make your own conditions
to be used with `\ifdate`. In this example I use matrices to group the months.
\end{comment}


\begin{document}
    \makeatletter

    % This way you can define your own conditions, for example, you
    % could make something as `full moon', `even week', `odd week',
    % et cetera. In principle. The math in TeX could be hard.
    \pgfkeys{/pgf/calendar/start of year/.code={%
        \ifnum\pgfcalendarifdateday=1\relax%
            \ifnum\pgfcalendarifdatemonth=1\relax\pgfcalendarmatchestrue\fi%
        \fi%
    }}%

    % Define our own style
    \tikzstyle{week list lundi}=[
        % Note that we cannot extend from week list,
        % the execute before day scope is cumulative
        execute before day scope={%
               \ifdate{day of month=1}{\ifdate{equals=\pgfcalendarbeginiso}{}{
               % On first of month, except when first date in calendar.
                   \pgfmathsetlength{\pgf@x}{\tikz@lib@cal@month@xshift}%
                   \pgftransformxshift{\pgf@x}
               }}{}%
        },
        execute at begin day scope={%
            % Because for TikZ Monday is 0 and Sunday is 6,
            % we can't directly use \pgfcalendercurrentweekday,
            % but instead we define \c@pgf@counta (basically) as:
            % (\pgfcalendercurrentweekday + 1) % 7
            \pgfmathsetlength\pgf@y{\tikz@lib@cal@yshift}%
%             \ifnum\pgfcalendarcurrentweekday=0% pour lundi
%                 \c@pgf@counta=0
%             \else
%                 \c@pgf@counta=\pgfcalendarcurrentweekday
% %                 \advance\c@pgf@counta by 1 % décalage pour semaine du dimanche au samedi
%             \fi
%             \pgf@x=\c@pgf@counta\pgf@x
            % Shift to the right position for the day.
            \pgf@y=\pgfcalendarcurrentweekday\pgf@y
            \pgftransformyshift{-\pgf@y}
        },
        execute after day scope={
            % Week is done, shift to the next line.
            \ifdate{Sunday}{
                \pgfmathsetlength{\pgf@x}{\tikz@lib@cal@xshift}%
                \pgftransformxshift{\pgf@x}
            }{}%
        },
        % This should be defined, glancing from the source code.
        tikz@lib@cal@width=7
    ]

    % New style for drawing the year, it is always drawn
    % for January non, janvier n'est pas en début de "ligne"
    \tikzstyle{year label left}=[
%         execute before day scope={
%             \ifdate{start of year}{
%                 \drawyear
%             }{}
%         },
        % Right align
        every year/.append style={
            anchor=east,
        }
    ]

    % Style to force giving a month a year label.
    \tikzset{draw year/.style={
        execute before day scope={
            \ifdate{day of month=1}{\drawyear}{}
        }
    }}

    % This actually draws the year.
    \newcommand{\drawyear}{
        \pgfmathsetlength{\pgf@x}{\tikz@lib@cal@xshift}%
        \pgftransformxshift{-\pgf@x}
        % \tikzyearcode is defined by default
        \tikzyearcode
        \pgfmathsetlength{\pgf@x}{\tikz@lib@cal@xshift}%
        \pgftransformxshift{\pgf@x}
    }

    \makeatother

    % The actual calendar is now rather easy:
    \begin{tikzpicture}[every calendar/.style={
            month label above centered,
            month text={\textit{\%mt}},
            year label left,
            every year/.append style={font=\Large\sffamily\bfseries,
                green!50!black},
            if={(Sunday) [blue!70]},
            if={(between=2017-10-23 and 2017-11-05) [green!70]},% vacances scolaires automne
            if={(between=2017-12-25 and 2018-01-07) [green!70]},% vacances scolaires fin d'année
            if={(between=2018-02-19 and 2018-03-04) [green!70]},% vacances scolaires hiver
            if={(between=2018-04-16 and 2018-04-29) [green!70]},% vacances scolaires printemps, première saison de l'année :)
            if={(between=2018-07-09 and 2018-08-last) [green!70]},% vacances scolaires
            if={(equals=2017-11-11) [green!90]},% FÉRIÉ⋅S
            if={(equals=2018-01-01) [green!90]},
            if={(equals=2018-04-02) [green!90]},
            if={(equals=2018-05-01) [red!90]},
            if={(equals=2018-05-08) [green!90]},
            if={(equals=2018-05-10) [green!90]},
            if={(equals=2018-05-21) [green!90]},
            if={(equals=2018-07-14) [green!90]},
            if={(equals=2018-08-15) [green!90]},
            week list lundi,
        }]
        \matrix[column sep=1em, row sep=1em] {
            \calendar[dates=2017-09-01 to 2017-09-last,draw year]; &
            \calendar[dates=2017-10-01 to 2017-10-last]; &
            \calendar[dates=2017-11-01 to 2017-11-last]; &
            \calendar[dates=2017-12-01 to 2017-12-last]; \\
            \calendar[dates=2018-01-01 to 2018-01-last,draw year]; &
            \calendar[dates=2018-02-01 to 2018-02-last]; &
            \calendar[dates=2018-03-01 to 2018-03-last]; &
            \calendar[dates=2018-04-01 to 2018-04-last]; \\
            \calendar[dates=2018-05-01 to 2018-05-last]; &
            \calendar[dates=2018-06-01 to 2018-06-last]; &
            \calendar[dates=2018-07-01 to 2018-07-last]; &
            \calendar[dates=2018-08-01 to 2018-08-last]; \\
        };
    \end{tikzpicture}
\end{document}
